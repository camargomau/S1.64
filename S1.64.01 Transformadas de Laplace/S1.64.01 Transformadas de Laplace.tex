\documentclass[12pt, a4paper]{article}
\usepackage[margin=1in]{geometry}
\usepackage{../auxiliary/utilities/preamble}

\newcommand{\titulo}{Transformadas de Lagrange}
\newcommand{\fecha}{11 de febrero de 2024}

\begin{document}
\sffamily
\begin{titlepage}
    \begin{center}
        \includegraphics[width=0.15\textwidth]{../auxiliary/assets/unam.png}
        \hspace{0.6\textwidth}
        \includegraphics[width=0.15\textwidth]{../auxiliary/assets/fes.png}

        \vspace*{5cm}
        \LARGE
        \textbf{\titulo}

        \vspace{1cm}
        \large
        Camargo Badillo Luis Mauricio \\
        \vspace{1.5cm}
        \textit{\fecha}

        \vfill

        \vspace{0.5cm}
        Ecuaciones Diferenciales II \\
        Oscar Gabriel Caballero Martínez \\
        Grupo 2602 \\
        \textbf{Matemáticas Aplicadas y Computación}\\
    \end{center}
\end{titlepage}


\newpage

\setcounter{section}{7}
\section{\texorpdfstring{\(f(t)=\cos(t)\)}{f (t) = cos (t)}}

Calculamos la transformada de Lagrange de \(f(t) = \cos(t)\):
\begin{align*}
	\mathcal{L}\{c\} &= \int_{0}^{\infty} e^{-st} \cos(t) \\
	&=
\end{align*}

\section{\texorpdfstring{\(f(t)=\sin(at)\)}{f (t) = sin (at)}}


\setcounter{section}{10}
\section{\texorpdfstring{\(f(t)=e^{4t}\)}{f (t) = e (4t) }}


\section{\texorpdfstring{\(f(t)=e^{-2t}\)}{f (t) = e (-2t)}}


\setcounter{section}{13}
\section{\texorpdfstring{\(f(t)=\sinh(3t)\)}{f (t) = sinh (3t)}}


\section{\texorpdfstring{\(f(t)=\cosh(6t)\)}{f (t) = cosh (6t)}}


\setcounter{section}{16}
\section{\texorpdfstring{\(f(t)=\cosh(at)\)}{f (t) = cosh (at)}}




\end{document}
